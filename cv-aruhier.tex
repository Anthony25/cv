\documentclass[11pt,a4paper,sans]{moderncv}

\moderncvstyle{classic}                             % style options are 'casual' (default), 'classic', 'oldstyle' and 'banking'
\moderncvcolor{blue}                               % color options 'blue' (default), 'orange', 'green', 'red', 'purple', 'grey' and 'black'

% this is a document holding multiple languages
% switch between ENGLISH and FRENCH by commenting one of the following lines:
\usepackage[french,english]{babel} % makes ENGLISH content
%\usepackage[english,french]{babel} % makes FRENCH content
% this is the macro to define phrases in two languages:
\newcommand{\babel}[2]{\ifnum\pdfstrcmp{\languagename}{english}=0 {#2}\else{#1}\fi}
\newcommand{\babelFR}[1]{\ifnum\pdfstrcmp{\languagename}{french}=0 {#1}\fi}
\newcommand{\babelEN}[1]{\ifnum\pdfstrcmp{\languagename}{english}=0 {#1}\fi}

\usepackage[utf8]{inputenc}
\usepackage{enumitem}

\usepackage[scale=0.75]{geometry}
\geometry{hmargin=1cm,vmargin=0.8cm}

\setlength{\hintscolumnwidth}{3.6cm}

% personal data
\firstname{Anthony}
\familyname{Ruhier}
\title{\textsc{Site Reliability Engineer}\newline{}\normalsize{Looking for an
internship in Software Engineering in the Bay area}}
\address{5 rue de Dublin}{}{90\,000 Belfort, France}
\mobile{%
    \babelFR{06~88~16~25~58}%
    \babelEN{+33~688~162~558}%
}
\homepage{aruhier.fr}
\email{anthony.ruhier@gmail.com}
\extrainfo{%
    \href{https://github.com/Anthony25}{https://github.com/Anthony25}
    \\
    \babelFR{23 ans (né le 22/10/1993), Français}%
    \babelEN{23 years old (born on 10/22/1993), French}%
}


%----------------------------------------------------------------------------------
%            content
%----------------------------------------------------------------------------------
\begin{document}

%-----       resume       ---------------------------------------------------------
\makecvtitle
\vspace{-3em}

    \section{%
        \babelFR{Expériences professionnelles}%
        \babelEN{Work experience}%
    }
\cventry{2014--2017}{\babelFR{Apprentis Ingénieur Développeur pour Opérateur Télécom} \babelEN{Apprentice Software Engineer for a local Telecom Operator}}{Trinaps}{}{\babelFR{Belfort (90000)}\babelEN{Belfort (France)}}{%
    \babelFR{Lead développeur, concevant des micro-services centrés sur
        l'automatisation de processus redondants, liés à la conception d'un
        Système d'Informations permettant à l'entreprise de efficacement ses
        données.}%
    \babelEN{Lead developer, designing micro-services focused on the automation
        of redundant processes, linked to the conception of an Information
        System allowing the company to efficiently store and use its data.}%
}

    \subsection{%
        \babelFR{Stages}%
        \babelEN{Internship}%
    }
\cventry{2014 (\babelFR{14 weeks}\babelEN{14 weeks})}{\babelFR{Développement d'un Extranet en Django}\babelEN{Intern developer of a customer dashboard in Django}}{Trinaps}{\babelFR{Belfort (90000)}\babelEN{Belfort (France)}}{}{}


    \section{%
        \babelFR{Diplômes}%
        \babelEN{Qualifications}%
    }
\cventry{2014--2017}{%
    \babelFR{Ingénieur en Informatique}%
    \babelEN{Master's Degree in C.S}%
}{UTBM}{%
    \babelFR{Ingénieur par Apprentissage}%
    \babelEN{Apprenticeship Program in 3 years}%
}{\babelFR{Belfort (90000)}\babelEN{Belfort (France)}}{}

\cventry{2012--2014}{%
    \babelFR{D.U.T Informatique}%
    \babelEN{Associate's degree in Software Engineering}%
}{IUT Belfort-Montbéliard}{}
{%
    \babelFR{Belfort (90000)}%
    \babelEN{Belfort (France)}%
}{}

    \section{%
        \babelFR{Projets}%
        \babelEN{Projects}%
    }

\cvitem{Personal infrastructure{}}
{\textbf{\babelFR{Infrastructure personnelle visant l'auto-hébergement}\babelEN{Personal infrastructure, to satisfy a will of self-hosting}}%
\babelFR{
    \begin{description}[nolistsep]
        \item[Web:] Nginx communiquant avec php-fpm ou uWSGI
        \item[SGBD:] PostgreSQL comme principal SGBD, MariaDB pour
            les applications qui le requièrent.
        \item[Hyperviseur:] LXC et KVM (Libvirt en tant qu'hyperviseur) sous
            ArchLinux, avec des sauvegardes automatiques vers un NAS Btrfs, via
            virt-backup et btrbk.
        \item [Orchestration:] Plus de 30 serveurs sur ArchLinux, gérés
            uniformément par Ansible, de la configuration du réseau aux bases
            de données, certificats X.509 et à la configuration d'Nginx.
        \item[DNS:] Serveurs d'autorités Knot avec signature DNSSEC
            automatique, résolveurs Unbound.
        \item [Réseau:] Les serveurs sont répartis sur 6 VLAN, via OpenVSwitch
            et un switch Cisco SG300. Le trafic est soumis à des règles de QoS,
            via PyQoS. Le trafic extérieur IPv4/IPv6 est dynamiquement routé
            par un serveur dédié, via OSPF et OpenVPN.
    \end{description}
}%
\babelEN{%
    \begin{description}[nolistsep]
        \item[Web:] Nginx communicating with php-fpm or uWSGI
        \item[Databases:] PostgreSQL as main DBMS, MariaDB for applications
            requiring it.
        \item[Hypervisor:] LXC and KVM (Libvirt as hypervisor) on ArchLinux,
            with automated backups to a Btrfs NAS, using virt-backup and btrbk.
        \item [Management:] More than 30 hosts running on ArchLinux, uniformly
            managed by Ansible from the network configuration to the
            databases, X.509 certificates and nginx setup.
        \item[DNS:] Authoritative servers use Knot with automatic DNSSEC
            signing, resolvers use Unbound.
        \item [Network:] Hosts are spread into 6 VLAN, via OpenVSwitch
            and a Cisco switch SG300. Traffic is subject to QoS policies, using
            PyQoS. IPv4/IPv6 external traffic is dynamically routed through a
            dedicated server, using OSPF and OpenVPN.
    \end{description}
}}
\vspace{-1em}

\cvitem{%
    PyQoS \newline{}
    \textit{\small
        Python \newline{}
        Traffic Control \newline{}
    }
}{\textbf{\babelFR{Framework Python de QoS sous Linux}\babelEN{Python
        framework to setup a QoS on Linux}
} (\textbf{\textcolor{color1}{\url{https://github.com/Anthony25/pyqos}}})
    \newline{}
    \babelFR{
        Framework pour construire des règles de QoS complexes pour Linux, de
        façon hiérarchiques tout en embarquant des optimisations prédéfinies.
        Les règles peuvent prendre avantage du système de POO de Python pour
        être le plus claires possible. Importé et utilisé à TRINAPS pour les
        installations évènementielles, pour gérer la QoS de plus de 5K
        utilisateurs simultanés, sur une bande passant WAN limitée (200mbps).
    }\babelEN{
        Framework to build complex QoS rules for Linux, in a hierarchical way
        with built-in tweaks. Rules can make use of the Python OOP system to be
        as much clear as possible. Imported and used at TRINAPS for events
        setups, to manage the QoS of more than 5K simultaneous users, with a
        really tight WAN bandwidth (200mbps).
}}

\cvitem{%
    \babelFR{Micro-services}\babelEN{Micro-services design} \newline{}
    \textit{\small
        RESTful API \newline{}
        RabbitMQ \newline{}
    }
}{\textbf{\babelFR{Design de micro-services pour TRINAPS}\babelEN{Design of
        TRINAPS' micro-services}}
    \newline{}
    \babelFR{
        Conception du design des nouveaux micro-services TRINAPS, devenant le
        nouveau standard interne. La communication inter-services est basée sur
        des API RESTful, documentée via Swagger, et sur des notifications via
        RabbitMQ\@. Les données sont versionnées via SQLAlchemy.
    }\babelEN{
        Conception of the new TRINAPS' micro-services design, become an
        internal standard. Cross-services communication is based on RESTful
        APIs, documented by Swagger, and notifications broadcasted by RabbitMQ.
        Data is versioned and handled by SQLAlchemy.
    }}

\cvitem{%
    \babelFR{Portail captif WiFi}\babelEN{WiFi Captive Portal} \newline{}
    \textit{\small
        Python (Flask) \newline{}
        Rust \newline{}
        Javascript \newline{}
    }
}{\textbf{\babelFR{Portail captif utilisé par TRINAPS en installation
évènementielle}\babelEN{Captive portal used by TRINAPS in events
installations}}
    \newline{}
    \babelFR{
        Principalement utilisé dans les festivals et évènementiels importants.
        Utilisé au festival des Eurockéennes depuis 2014, avec 3K utilisateurs
        connectés. Le portail communique avec un routeur sous Debian, gérant
        une hashmap d'adresses MAC stockées efficacement dans ipset.
    }\babelEN{
        Mainly used for festivals and important events.  Used in the
        Eurockeennes festival since 2014, with 3K connected users. The portal
        communicates with a router running on Debian, managing a hashmap of MAC
        addresses efficiently stored in ipset.
    }}
\vspace{-1.5em}


    \section{%
        \babelFR{Programmation}%
        \babelEN{Programming}%
    }

\cvitem{\babelFR{Impérative}\babelEN{Imperative Programing}}{Python (TDD), Rust, Java (TDD), C, C++}

\cvitem{\babelFR{Web}\babelEN{Web development}}{Django, Flask, Javascript}


\cvitem{\babelFR{Autres}\babelEN{Others}}{Bash, RabbitMQ, Docker}


    \section{%
        \babelFR{Compétences linguistiques}%
        \babelEN{Language skills}%
    }
\cvitem{\babelFR{Anglais}\babelEN{English}}{\textbf{\babelFR{Niveau C1}\babelEN{C1 Level}}, BULATS -- 85/100 (2016)}
\babelEN{\cvitem{French}{\textbf{Native speaker}}}
\cvitem{\babelFR{Espagnol}\babelEN{Spanish}}{\textbf{\babelFR{Niveau scolaire}\babelEN{Basic}}}
\end{document}

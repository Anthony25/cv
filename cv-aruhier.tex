\documentclass[11pt,a4paper,sans]{moderncv}

\moderncvstyle{classic}                             % style options are 'casual' (default), 'classic', 'oldstyle' and 'banking'
\moderncvcolor{blue}                               % color options 'blue' (default), 'orange', 'green', 'red', 'purple', 'grey' and 'black'

% this is a document holding multiple languages
% switch between ENGLISH and FRENCH by commenting one of the following lines:
\usepackage[french,english]{babel} % makes ENGLISH content
%\usepackage[english,french]{babel} % makes FRENCH content
% this is the macro to define phrases in two languages:
\newcommand{\babel}[2]{\ifnum\pdfstrcmp{\languagename}{english}=0 {#2}\else{#1}\fi}
\newcommand{\babelFR}[1]{\ifnum\pdfstrcmp{\languagename}{french}=0 {#1}\fi}
\newcommand{\babelEN}[1]{\ifnum\pdfstrcmp{\languagename}{english}=0 {#1}\fi}

\usepackage[T1]{fontenc}
\usepackage{lmodern}
\usepackage[utf8]{inputenc}
\usepackage{enumitem}

\usepackage[scale=0.75]{geometry}
\geometry{hmargin=1cm,vmargin=0.8cm}

\setlength{\hintscolumnwidth}{3.6cm}

% personal data
\firstname{Anthony}
\familyname{Ruhier}
\title{\textsc{Infrastructure Software Engineer}}
\address{17 rue de la Libération}{}{25\,410 Pouilley-Français, France}
\mobile{%
    \babelFR{06~88~16~25~58}%
    \babelEN{+33~688~162~558}%
}
\homepage{aruhier.fr}
\email{anthony.ruhier@gmail.com}
\extrainfo{%
    \href{https://github.com/aruhier}{https://github.com/aruhier}
    \\
    \babelFR{25 ans (né le 22/10/1993), Français}%
    \babelEN{25 years old (born on 10/22/1993), French}%
}


%----------------------------------------------------------------------------------
%            content
%----------------------------------------------------------------------------------
\begin{document}

%-----       resume       ---------------------------------------------------------
\makecvtitle
\vspace{-3em}

    \section{%
        \babelFR{Expériences professionnelles}%
        \babelEN{Work experience}%
    }
\cventry{2019}{Network Engineer, Software and Automation on CorpNet}{Google}{}{\babelFR{Zürich (Suisse)}\babelEN{Zürich (Switzerland)}}{%
    \babelFR{Ingénieur réseau focalisé sur le développement et
        l'automatisation, travaillant dans l'équipe Corporate and Entreprise
        Infrastructure pour délivrer la nouvelle génération du réseau et
        Data Centers Corp de Google. Développe des logiciels au sein de services
        distribués, abstractions et composants de systèmes étant au cœur
        de Google.}%
    \babelEN{As a Network Engineer focused on software and automation, I work
        on the Corporate and Enterprise infrastructure team to deliver Google's
        next generation network and corporate data centers. I build software for
        distributed services, abstractions and the components of the system that
        operates and powers Google.}%
}
\cventry{2018}{Network DevOps Engineer}{Online/Scaleway}{}{\babelFR{Paris (75116)}\babelEN{Paris (France)}}{%
    \babelFR{Travaille avec les SRE backbone sur le maintien et l'amélioration
        du parc, en leur concevant et développant des outils --- principalement
        en Python --- qui leur permettent ensuite de paralléliser et
        automatiser leur actions visant plusieurs milliers d'utilisateurs.}%
    \babelEN{Working with the Backbone SREs on maintaining and improving the
        entire network, by designing and coding tools --- mostly in Python --- to allow
        them to parallelize and automate their actions on a larger scale, to target
        a hundred of thousands of users. }%
}
\cventry{2014--2017}{\babelFR{Apprenti Ingénieur Développeur pour Opérateur Télécom}\babelEN{Apprentice Software Engineer for a local Telecom Operator}}{Trinaps}{}{\babelFR{Belfort (90000)}\babelEN{Belfort (France)}}{%
    \babelFR{Lead développeur, concevant des micro-services centrés sur
        l'automatisation de processus redondants, liés à la conception d'un Système
        d'Informations permettant à l'entreprise de gérer efficacement ses données.}%
    \babelEN{Lead developer, designing micro-services focused on the automation
        of redundant processes, linked to the conception of an Information
        System allowing the company to efficiently store and use its data.}%
}


    \section{%
        \babelFR{Diplômes}%
        \babelEN{Qualifications}%
    }
\cventry{2014--2017}{%
    \babelFR{Ingénieur en Informatique}%
    \babelEN{Master's Degree in C.S}%
}{UTBM}{%
    \babelFR{Ingénieur par Apprentissage}%
    \babelEN{Apprenticeship Program in 3 years}%
}{\babelFR{Belfort (90000)}\babelEN{Belfort (France)}}{}


    \section{%
        \babelFR{Projets}%
        \babelEN{Projects}%
    }

\cvitem{\babelFR{Projet personnel}\babelEN{Personal project}\newline{}}
{\textbf{\babelFR{Infrastructure personnelle visant l'auto-hébergement}\babelEN{Personal infrastructure, to satisfy a will of self-hosting}}%
\babelFR{
    \begin{description}[nolistsep]
        \item[Web:] Nginx comme ingress controller pour Kubernetes.
        \item[SGBD:] PostgreSQL comme principal SGBD, MariaDB pour
            les applications qui le requièrent.
        \item[Hyperviseur:] 3 nœud Kubernetes sur KVM (Libvirt en tant
            qu'hyperviseur) sous Debian, avec des sauvegardes automatiques vers
            des NAS Glusterfs ZFS et Btrfs via virt-backup et btrbk.
        \item [Orchestration:] Plus de 30 pods et serveurs sous ArchLinux, gérés
            uniformément par Ansible, de la configuration du réseau aux bases
            de données, Nginx et certificats X.509.
        \item[DNS:] Serveurs d'autorités Knot avec signature DNSSEC
            automatique, résolveurs Unbound.
        \item [Réseau:] Les serveurs sont répartis sur 6 VLAN, via OpenVSwitch
            et un switch Cisco SG300. Le trafic est soumis à des règles de QoS,
            via PyQoS. Le trafic extérieur IPv4/IPv6 est dynamiquement routé
            par un serveur dédié, via BGP et Wireguard. Résolveurs DNS en
            anycast.
    \end{description}
}%
\babelEN{%
    \begin{description}[nolistsep]
        \item[Web:] Nginx as ingress controller in Kubernetes.
        \item[Databases:] PostgreSQL as main DBMS, MariaDB for applications
            requiring it.
        \item[Hypervisor:] 3 Kubernetes nodes over KVM (Libvirt as hypervisor)
            on Debian, with automated backups to Btrfs and ZFS Glusterfs NAS,
            using virt-backup and btrbk.
        \item [Management:] More than 30 pods and hosts running on ArchLinux,
            uniformly managed by Ansible from the network configuration to the
            databases, X.509 certificates and Nginx setup.
        \item[DNS:] Authoritative servers use Knot with automatic DNSSEC
            signing, resolvers use Unbound.
        \item [Network:] Hosts are spread into 6 VLAN, via OpenVSwitch and a
            Cisco switch SG300. Traffic is subject to QoS policies, using
            PyQoS. IPv4/IPv6 external traffic is dynamically routed using BGP
            and Wireguard. DNS resolvers in anycast.
    \end{description}
}}
\vspace{-1em}

\cvitem{%
    Online/Scaleway \newline{}
    \textit{\small
        Python \newline{}
        Celery \newline{}
    }
}{\textbf{\babelFR{Rate limiting de l'entièreté du parc de
    serveurs}\babelEN{Rate limiting the entire server farm uplinks}
}
    \newline{}
    \babelFR{
        Mise en place d'un rate limiting dynamique sur plus de 3K switches
        maison d'Online, variant suivant l'offre du client, avec une taille de
        burst calculée suffisamment large pour être le moins contraignant
        possible. Impacte plus de 50K serveurs, a permis d'économiser plusieurs
        100gbps de bande passante.
    }\babelEN{
        Dynamically set rate limiting on 3K Online's custom switches, depending
        on the customer offer, with a computed burst size large enough to be
        the least restrictive possible. Impacted more than 50K servers, saved
        several 100gbps of bandwidth.
}}

\cvitem{%
    \babelFR{Projet personnel}\babelEN{Personal project}\newline{}
    \textit{\small
        Python \newline{}
        Traffic Control \newline{}
    }
}{\textbf{\babelFR{Framework Python de QoS sous Linux}\babelEN{Python
        framework to setup a QoS on Linux}
} (\textbf{\textcolor{color1}{\url{https://github.com/aruhier/pyqos}}})
    \newline{}
    \babelFR{
        Framework pour construire des règles de QoS complexes pour Linux, de
        façon hiérarchiques tout en embarquant des optimisations prédéfinies.
        Les règles peuvent prendre avantage du système de POO de Python pour
        être le plus clair possible. Importé et utilisé à TRINAPS sur les
        installations évènementielles, pour gérer la QoS de plus de 5K
        utilisateurs simultanés, sur une bande passante WAN limitée (200mbps).
    }\babelEN{
        Framework to build complex QoS rules for Linux, in a hierarchical way
        with built-in tweaks. Rules can make use of the Python OOP system to be
        as much clear as possible. Imported and used at TRINAPS for events
        setups, to manage the QoS of more than 5K simultaneous users, with a
        really tight WAN bandwidth (200mbps).
}}


\cvitem{%
    \babelFR{TRINAPS}\babelEN{TRINAPS} \newline{}
    \textit{\small
        Python (Flask) \newline{}
        Rust \newline{}
        Javascript \newline{}
    }
}{\textbf{\babelFR{Portail captif utilisé par TRINAPS en installation
évènementielle}\babelEN{Captive portal used by TRINAPS in events
installations}}
    \newline{}
    \babelFR{
        Principalement utilisé dans les festivals et évènementiels importants.
        Utilisé au festival des Eurockéennes depuis 2014, avec 3K utilisateurs
        connectés. Le portail communique avec un routeur sous Debian, gérant
        une hashmap d'adresses MAC stockées efficacement dans ipset.
    }\babelEN{
        Mainly used for festivals and important events.  Used in the
        Eurockeennes festival since 2014, with 3K connected users. The portal
        communicates with a router running on Debian, managing a hashmap of MAC
        addresses efficiently stored in ipset.
    }}
\vspace{-1.5em}


    \section{%
        \babelFR{Programmation}%
        \babelEN{Programming}%
    }

\cvitem{\babelFR{Impérative}\babelEN{Imperative Programing}}{Python (TDD), Rust, Java (TDD), C, C++}

\cvitem{\babelFR{Web}\babelEN{Web development}}{Django, Flask, Javascript}


\cvitem{\babelFR{Autres}\babelEN{Others}}{
    Bash, RabbitMQ, Docker
}


    \section{%
        \babelFR{Compétences linguistiques}%
        \babelEN{Language skills}%
    }
\cvitem{\babelFR{Anglais}\babelEN{English}}{\textbf{\babelFR{Niveau C1}\babelEN{C1 Level}}, BULATS -- 85/100 (2016)}
\babelEN{\cvitem{French}{\textbf{Native speaker}}}
\end{document}
